\documentclass[12pt,a4paper]{report}
\usepackage[utf8]{inputenc}
\usepackage[spanish]{babel}
\usepackage{amsmath}
\usepackage{amsfonts}
\usepackage{amssymb}
\usepackage{graphicx}
\usepackage[left=2cm,right=2cm,top=2cm,bottom=2cm]{geometry}
\usepackage{listings} %Inserta código en el archivo
\usepackage{graphicx} %Inserta imagenes en el archivo
%\usepackage{cite} % para contraer referencias
%\usepackage{lscape} %cambia la horientación de la página a horizontal
\usepackage{hyperref} %permite insertar direcciones web

%\usepackage{fullpage}


\begin{document}

\title{Desarrollo de un software hidrosedimentario como herramienta open-source para el análisis de los ríos en el Perú}
\author{Villanueva Portella, Jhon Gesell}
\date{18/03/2019}
\maketitle
\section{Relevacia y problema de la investigación}
Un paquete hidro-sedimentario ayuda al entendimiento de la mecánica de fluidos en la interacción que llega a existir entre un afluente y los sedimentos en suspención que se encuentran en él, los datos que se podrían procesar permitirían la pronta acción por parte de las autoridades pertinentes para prevenir riesgos debido a las altas precipitaciones que llegan a haber en épocas de máximas avenidas; en el norte del Perú en los departamentos de Piura y Tumbes es muy usual que en las épocas de máximas avenidas ocurran inundaciones en los centros poblados aguas abajo; la data histórica hidrológica en muchos apartados del interior del país no se ha gestionado bien, debido a ello los especialistas técnicos no logran del todo hacer buenos estudios definitivos para los proyectos de desarrollo urbano.

El poder almacenar correctamente la información es una necesidad principal para que de manera permanente se encuentre la información de primera mano a disposición del cuerpo de ingenieros que estudian la hidrología y cambio climático.

Las tecnologías hoy disponibles en el mercado se hacen inaccesibles por la insuficiencia de fondos para la renovación de licencias de los softwares que estos a su vez tienen un estado de dependencia a sistemas operativos privativos evitando de esta manera que tanto municipalidades, centros de investigación y centros de estudios superiores no puedan participar en el estudio y comportamiento de los ríos en el Perú. Se debe entender también que hasta la fecha de esta investigación solamente se llegó a identificar un software que lograba esta tarea de gestionar la información e imprimir resultados gráficos, aunque su desarrollador ya lo ha discontinuado, los alcances que esté presentaba eran de almacenar los datos en una base de datos para el gestor de Microsoft Office Access, por ello al día de hoy lo que se encuentra en el mercado son software que permiten visualizar resultados más no almacenar la información, además que su comportamiento de estos es la de una caja negra ya que no permiten ver el código fuente del programa de computadora restringido unicamente a un sistema operativo.

\section{Hipótesis}
	El software que se propone desarrollar a través de la investigación permitirá almacenar datos recolectados mediante el ADCP Rio Grande de 1200 kHz y las muestras de sedimentos en suspención procesadas para finalmente mostrar resultados gráficos superpuestos de todas las celdas en escala de colores para la sección del río con los valores de la velocidad en coordenadas cartesianas, puntos de perfiles de concentración de sedimentos en diferentes profundides para cada punto de muestreo en el transecto, calculo del caudal.
\section{Objetivos de la investigación}
	\subsection{Ojetivos Generales}
	\begin{itemize}
	\item Crear un software hidrosedimentario con interfaz gráfica de usuario.
	\item Brindar un producto que ayude a los científicos e ingenieros que trabajan con fluídos geofísicos.
	\item Entregar un producto open-source para la comunidad científica internacional.
	\end{itemize}
	\subsection{Objetivos Específicos}
	\begin{itemize}
	\item Crear una lectura de archivos de caudales.
	\item Crear un formulario para insertar los datos de las muestras se sedimentos.
	\item Crear una base de datos.
	\item Gráficar la sección del río para visualizar las cotas de fondo y los valores de sedimentos en suspensión.
	\end{itemize}
\section{Metodología}
La investigación obedecerá al siguiente flujo de trabajo:
\begin{enumerate}
\item Afora en la estación hidrológica El Tigre: El fin es entender la forma de trabajo y los riesgos que llegan a tener para tomar las muestras las personas que aforan el río.
\item Procesamiento de muestras en el laboratorio de agua y suelos: Se hace el filtrado de las muestras de sedimentos en rampas que cuentan con bombas de vacío y estas son metidas a la estufa para su secado y posterior pesado, finalmente toda la información es ordenada en una tabla impresa en papel.
\item Los datos recopilados por el ADCP Rio Grande de 1200 kHz son ubicados en el disco duro de una computadora según una jerarquía de carpetas que debe respetarse.
\item Se desarrollan mokaups tentativos para identificar todas las ventanas, menús y widgets con los que contará el software.
\item Mediante el framework Qt Designer se crean los archivos GUI con extensión .ui que serán llamados posteriormente para dar funcionalidad al software en el backend.
\item Se crea un nuevo archivo con la extensión del lenguaje de programación Python desde la cual se importan las librerías de PyQt5 y se importa el script de la GUI que había sido desarrollada gracias a Qt5 Designer, se dá funcionalidad a todos los objetos, se generan el esquema para la base de datos y otros componentes.
\item Se hace la compilación para los sistemas operativos Windows y Ubuntu mediante la librería Pyinstaller con el cual ya se podrá contar con un programa ejecutable, bastará con hacer doble clic en el programa para que este se abra y comience a trabajar el usuario.
\item Se crea la Guía de Usuario y se sube el código fuente a un respositorio en Github.
\end{enumerate}




\section{Bibliografía}
\begin{itemize}
	\item Gonzáles, R. (s.f.). \textit{Python para todos}. Recuperado de: \textbf{http://mundogeek.net/tutorial-python}
	\item Coutinho, N. (2016). \textit{Introducción a la programación con Python: Algoritmos y lógica de programación para principiantes}, Brasil: Novatec Editora Ltda.
	\item Harwani, B. (2018). \textit{Qt5 Python GUI Programming Cookbook: Building responsive and powerful cross-platform applications with PyQt}, Estados Unidos: Packt Publishing Ltd.
	\item Owens, M. and Allen, G. (2010). \textit{The Definitive Guide to SQLite}, Estados Unidos: Springer.
	\item Johansson, R. (2015). \textit{Numerical Python}, Estados Unidos: Springer
	\item Carneiro, M. (2007). \textit{Manual de redacción superior}, Perú: Editorial San Marcos E.I.R.L.
\end{itemize}


\end{document}