\documentclass[12pt,a4paper]{report}
\usepackage[utf8]{inputenc}
\usepackage[spanish]{babel}
\usepackage{amsmath}
\usepackage{amsfonts}
\usepackage{amssymb}
\usepackage{graphicx}
\usepackage[left=2cm,right=2cm,top=2cm,bottom=2cm]{geometry}
\usepackage{listings} %Inserta código en el archivo
\usepackage{graphicx} %Inserta imagenes en el archivo
%\usepackage{cite} % para contraer referencias
%\usepackage{lscape} %cambia la horientación de la página a horizontal
\usepackage{hyperref} %permite insertar direcciones web


\usepackage{fullpage}

\begin{document}

\title{Plan de tesis}
\author{Villanueva Portella, Jhon Gesell}
\date{18/03/2019}
\maketitle

\section{Introducci\'on}
	\begin{itemize}
	\item Importancia
	\item Antecedentes
	\item Justificaci\'on del m\'etodo utilizado
	\item Valor agregado de nuestra propuesta	
	\end{itemize}
Un software hidro-sedimentario ayudará a que las personas puedan tomar mejores decisiones en la gesti\'on de los recursos h\'idricos y as\'i a la vez evitar que en \'epocas de m\'aximas avenidas ocurran desastres humanos, con esta herramienta lo que se busca constribuir a la comunidad cient\'ifica haciendo posible su acceso a todo el mundo.

En el pasado ya se ha contado con herramientas privativas o en otros casos con programas no muy intuitivos para los usuarios finales.

El m\'etodo utilizado es de sobre posición pasición de valores en una matríz raíz que representa los valores de posición en las coordenadas x, y y una tercera componente que es de de la velocidad.

El software que se propone esta a la altura de otras herramientas con unas posibilidades abiertas de ser alojada en una plataforma web y de esta manera hacer computaci\'on en la nube con sistemas embebidos, por ahora buscamos hacer computaci\'on en un localhost que brinde posibilidades de acceso desde centros de estudio, centros de investigaci\'on o incluso para el sector privado; las herramientas que muchas veces podemos encontrar en el mercado tienen un costo que no permite acceder a ellas y por ser de consideraci\'on a los pa\'ises en v\'ias de desarrollo nuestra propuesta pretende beneficiar a ellos.
	\subsection{Ojetivo General}
	\begin{itemize}
	\item Crear un software hidrosedimentario con interfaz gr\'afica de usuario.
	\item Brindar un producto que ayude a los cient\'ificos e ingenieros que trabajan con flu\'idos geof\'isicos.
	\item Entregar un producto open-source para la comunidad cient\'ifica internacional.
	\end{itemize}
	\subsection{Objetivo Espec\'ifico}
	\begin{itemize}
	\item Leer los archivos ASCII que nos brinda el equipo ADCP (Acoustic Doppler Current Profiler).
	\item Crear un formulario para insertar los datos de las muestras se sedimentos.
	\item Crear una base de datos.
	\item Graficar la sección del r\'io para visualizar las cotas de fondo y los valores de sedimentos en suspensi\'on.
	\item Crear
	\end{itemize}

	\subsection{Hip\'otesis}

\section{Materiales y m\'etodos}
\section{Resultados esperados}
\section{Cronograma}
\section{Discusiones}
\section{Conclusiones}
\section{Bibliograf\'ia}
\end{document}